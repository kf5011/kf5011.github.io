\documentclass[10pt, hyperref={pdfpagelabels=false}]{beamer}

\usepackage{tikz, verbatim, enumitem}

\usetikzlibrary{decorations}
\usetikzlibrary{backgrounds}
\usetikzlibrary{patterns}
\usetikzlibrary{snakes}
\usetikzlibrary{shapes}
\usetikzlibrary{positioning, shapes.geometric, arrows.meta}
\usetikzlibrary{arrows,automata}

\setlength{\parindent}{0pt}
\setlength{\parskip}{1.3ex}

\title{Encryption:\\The RSA Public Key Cipher}
\author{Michael Brockway}
\date{\today}

\setlist[enumerate]{itemsep=0mm}
\setitemize{label=\usebeamerfont*{itemize item}
  \usebeamercolor[fg]{itemize item}
  \usebeamertemplate{itemize item}}

\begin{document}

\begin{frame}
\titlepage
\end{frame}

\begin{frame}
\frametitle{Overview}
Transport-layer security employs an asymmetric public cryptosystem to allow two parties (usually a client application and a server) to authenticate each other and negogiate parameters for their conversation.

This lecture describes RSA, the cryptosystem devised by Rivest, Shamir and Adleman and commonly used for this putpose.

RSA makes heavy use of \emph{modular arithmetic}; we start start with a reminder about this and an overiview of some key tools and principles needed for RSA.
\end{frame}

\begin{frame}
\frametitle{Modular Arithmetic}
In C/C++/Java, if \texttt{\color{blue}a}, \texttt{\color{blue}n} are \texttt{\color{blue}int} variables, the expression \texttt{\color{blue}a \% n} denotes the \emph{remainder} when \texttt{\color{blue}a} is integer-divided by \texttt{\color{blue}n}.

In more mathematical language, we can say \emph{\color{blue}a, b are congruent} (or \emph{\color{blue}equivalent}) \emph{\color{blue}modulo n}, written {\color{blue}$a \equiv_n b$} to mean that a, b have the same remainder on integer-division by n: $a \% n = b \%n$.

Modulo $n$ arithmetic features integers 0, 1, 2, 3, ... n-1. When you count up to $n$ you wrap around to 0. Instead of a `number line'
\begin{figure}
\includegraphics[width=\textwidth]{numLine.png}
\end{figure}
\end{frame}

\begin{frame}
\frametitle{Modular Arithmetic - plus and minus}
... we have a `number circle': here is one for modulo-15:
\begin{figure}
\includegraphics[width=\textwidth]{numCircle15.png}
\end{figure}

$(5 + 12) \equiv_{15} 17 \equiv_{15} 2$;
\begin{itemize}
\item ... or count around the circle: start from 5, count on 12, end up on 2.
\end{itemize}

$3 - 8 \equiv_{15} 10$
\begin{itemize}
\item start from 3, count back 8.
\item 0r if you prefer, $(3 - 8) \equiv_{15} (15 + 3 - 8) \equiv_{15} 10$.
\end{itemize}
\end{frame}

\begin{frame}
\frametitle{Modular Arithmetic - multiplication}
We can multiply: $8 \times 5 \equiv_{15} 40 \equiv_{15} 10$. Check that starting from 0
and counting 8 lots of 5 around the circle, you end up at 10.

\begin{figure}
\caption{Modulo 15 multiplication table}
\includegraphics[width=0.8\textwidth]{mod15mulTbl.png}
\end{figure}
\end{frame}

\begin{frame}
\frametitle{Modular Arithmetic - reciprocals}
It is possible to divide, \emph{sometimes}. Interestingly 5 has no multiplicative
inverse, or reciprocal modulo 15 (is there a 1 in row 5 of the multiplication table?)
 but 8 does: there is a 1  in row 8, in column 2; so 2 is the mod 15 reciprocal of 8 (and \emph{vice versa}).

$x$ has a {\color{blue}reciprocal modulo $n$} (a number $y$ such that $x \times y \equiv_n 1$) if and only if $x$ \emph{has no factor} (other than 1) in common with $n$. ($x$ and $n$ are \emph{relatively prime}).

Since 5 has a factor in common with 15, 5 has no reciprocal modulo 15.

The reciprocal of $x$ modulo $n$ we write = $x^{-1} mod~n$.

Rows 0, 3, 5, 6, 9, 10, 12 of modulo 15 multiplication tabledo not have a 1 in. These numbers have a factor in common with 15, and so do not have a reciprocal modulo 15.

\begin{itemize}
\item 2 has a reciprocal: 8 ($2 \times 8 \equiv_{15} 1$ so $2^{-1} mod~15 = 8$)
\item 13 has reciprocal, 7
\item 4 is its own reciprocal. So is 11.
\end{itemize}
\end{frame}

\begin{frame}
\frametitle{Modular Arithmetic - reciprocals}
When the modulus $n$ is a \emph{\color{blue}prime} number all numbers other than 0 have a reciprocal modulo $n$. Here is the modulo-13 multiplication table:

\begin{figure}
\includegraphics[width=0.9\textwidth]{mod13mulTbl.png}
\end{figure}
\end{frame}

\begin{frame}
\frametitle{Modular Arithmetic - exponentials}
We also need to do exponentiation:
\begin{itemize}
\item $3^4 = 3 \times 3 \times 3 \times 3 = 81 \equiv_{15} 6$
\item $3^4 = 81 \equiv_{9} 0$
\item $5^6 = 5 \times 5 \times 5 \times 5 \times 5 \times 5\equiv_{9} 25 \times 25 \times 25 \equiv_{9} 7 \times 7 \times 7$ \\
{\color{brown}[because $25 \equiv_{9} 7$]} $\equiv_{9} 4 \times 7 \equiv_{9} 28 \equiv_{9} 1$
\end{itemize}

Here is a table of exponentials in modulo 9, for nonzero exponent numbers
\begin{figure}
\includegraphics[width=0.5\textwidth]{mod9eptls.png}
\end{figure}
\end{frame}

\begin{frame} [fragile]
\frametitle{Modular Arithmetic - exponentiation}

The algorithm to calculate b to the power of e, modulo n is quite neat:
\color{blue}
\begin{verbatim}
set result initially = 1;
while (e != 0) {
  if (e modulo 2 != 0)
    set result = (result * b) mod n;
  set e = e / 2;
  set b = (b*b) mod n;
  }
return result;
\end{verbatim}
\end{frame}

\begin{frame}
\frametitle{Modular Arithmetic - tools}
You will find a number of resources for exploring modular arithmetic at 
\texttt{\small\color{blue}http://computing.northumbria.ac.uk/staff/cgmb3/teaching/\\cryptography/index\_crypto.html}

In particular, 
\begin{itemize}
\item \href{http://computing.northumbria.ac.uk/staff/cgmb3/teaching/cryptography/ModArithTables.xls}{\textcolor{blue}{Spreadsheets}}[click] from which the the multiplication and exponentiation tables above were made. Add your own sheets!
\item \href{http://computing.northumbria.ac.uk/staff/cgmb3/teaching/cryptography/ModularCalc.jar}{\textcolor{blue}{A modular calculator}}[click] - run at a command-line, eg\\ \texttt{\color{brown}java -jar ModularCalc.jar 15} 
\item \href{http://computing.northumbria.ac.uk/staff/cgmb3/teaching/cryptography/ModExp/ModExpAplt.class}{\textcolor{blue}{A java application to calculate modular exponentials}}[click] - download and run with java.
\item \href{http://computing.northumbria.ac.uk/staff/cgmb3/teaching/cryptography/EuclideanAlg/EuclideanAlg.class}{\textcolor{blue}{A java implmentation of the extended Euclidean algorithm}}[click] - download and run with java.
\end{itemize}
This web page has links to the java source code of these tools, for your interest.
\end{frame}

\begin{frame}
\frametitle{Modular Arithmetic - finding reciprocals}
Computationally, modular reciprocals are found using the Euclidean algorithm
\begin{itemize}
\item See the last link above.
\item Given two numbers x, y this algorithm will will find their \emph{greatest common divisor} – the largest number that exactly divides into x and y.
\item It also displays the GCD as a linear combination of x, y:
  \begin{itemize}
  \item It will find r, s such that $GCD(x, y) = rx + sy$
  \end{itemize}
\item Eg $GCD(24, 54) = 6$; $6 = 24(-2) + 54 \times 1$
\item For $x^{-1} mod~n$ to exist, $x$ and $n$ have to be relatively prime: $GCD(x, n) = 1$.
\end{itemize}
\end{frame}

\begin{frame}
\frametitle{Modular Arithmetic - finding reciprocals}
To calculate $x^{-1}$ mod $n$,
\begin{itemize}
\item Feed x, n into the Euclidean algorithm: get out numbers r, s such that
$1 = GCD (x, n) = r x + s n$
\item This tells us that $r x \equiv_n 1$: in other words, $x^{-1} mod~n$ is $r$.
\end{itemize}

For example $200003^-1 mod~263160$:
\begin{itemize}
\item feed 200003 and 263160 into the Euclidean algorithm and we get
a message that their GCD = 1 and $1 = -68573\times263160 + 90227\times200003$
\item This tells us $200003^{-1}$ mod 263160 = 90227
\end{itemize}
\end{frame}

\begin{frame} [fragile]
\frametitle{Euclidean algorithm in brief}
The ``classic'' Euclidean algorithm:
{\color{blue}
\begin{verbatim}
To find the GCD of x and y {
repeat {
  Subtract the smaller of x, y from the bigger
  Replace the bigger of x, y by this number
}
...until 0 is obtained.
The last nonzero number is the GCD
}
\end{verbatim}
}

Example: GCD(36, 102):
\begin{itemize}
\item $36,102 \rightarrow 66; 36,66 \rightarrow 30; 36,30 \rightarrow 6; 30,6 \rightarrow 24;$ $24,6 \rightarrow 18; 18,6 \rightarrow 12; 12,6 \rightarrow 6; 6,6 \rightarrow 0$.
\item GCD = 6
\end{itemize}
\end{frame}

\begin{frame}
\frametitle{Euclidean algorithm in brief}
To recover r and s such that $GCD(x, y) = r x + s  y$
\begin{itemize}
\item It is necessary to keep track of all the repeated subtractions
\item ``Unwind'' them to recover the original x, y; build up the values of
r, s as you go.
\end{itemize}

The Euclidean algorithm application uses integer division rather than subtraction to speed things up a bit.

x, y $\rightarrow$ quotient q, remainder r
\begin{itemize}
\item subtract x from y repeatedly until a number smaller than x is left
\item q counts the subtractions, r is the number left
\item $q = y / x$, $r = y~\%~x$
\end{itemize}
\end{frame}

\begin{frame}
\frametitle{RSA - Rivest, Shamir, Adleman Public Key Cryptosystem}
Based on a clever theorem -
\begin{itemize}
\item Pierre de Fermat, in the 1600s, showed that if $m$ is any integer whatever, and $p$ is a prime number then {\color{blue}$m^{(p-1)} \equiv_p 1$}.
\item Eg $4^6 = 8^3 \equiv_7 1^3 = 1$
\item Leonard Euler, in the 1700s, extended this result: If $p$ and $q$ are prime numbers, and $n = p q$, and $m$ is any number relatively prime to $n$, then $m^{(p-1)(q-1)} \equiv_n 1$.
\item It follows that for any $k$, {\color{blue}$m^{k(p-1)(q-1)+1} \equiv_n m$}.
\end{itemize}

To use this for cryptography,
\begin{itemize}
\item Our $m$ will be a block of bits -- part of a message to be encrypted
\item The operation of raising $m$ to the power of $(p-1)(q-1) + 1$ (mod n)
we will split into 2 parts: part 1 will be the encryption process, part
2 will be the decryption process; applying one part after the other
will give back the message block we started with.
\end{itemize}
\end{frame}

\begin{frame}
\frametitle{Definition of an RSA Cryptosystem}
\begin{itemize}
\item Think of two big prime numbers -- nowadays, of the order of $2^{1024}$; call them p, q; Compute $n = p q$
\item Pick a random number d bigger than p and q but smaller than n, such that d is relatively prime to $(p-1)(q-1)$
\item Compute e, the reciprocal of d mod $(p-1)(q-1)$.
  \begin{itemize}
  \item $e d~\%~(p-1)(q-1) = 1$ (ie $e d \equiv_{(p-1)(q-1)} 1$)
  \end{itemize}
\item Publish the pair n, e -- these are your public key. Keep p, q, d secret
\item Encryption: from each block of bits m, compute $c = m^e~\%~n$
\item Decryption: from each encrypted block c, compute $m = c^d~\%~n$
\end{itemize}
\end{frame}

\begin{frame}
\frametitle{Why this works}
From Euler’s theorem, $m^{k(p-1)(q-1)+1} \equiv_n m$

But we made e and d so that $e d \equiv_{(p-1)(q-1)} 1$. That is, e.d = 1 + a multiple of (p-1)(q-1). 

So Euler’s result tells us that $m^{ed} \equiv_n m$

In other words, $(m^e)^d \equiv_n m$

So if $c \equiv_n m^e$ then $c^d \equiv_n m$

In other words, if we encrypt by computing $c = m^e~\%~n$, then computing $c^d~\%~n$
recovers the original $m$

Thus encryption is {\color{blue}$c = m^e~\%~n$}; decryption is {\color{blue}$m = c^d~\%~n$}.
\end{frame}

\begin{frame}
\frametitle{RSA example}
$2^{18} = 262144$; With an $n$ just above this we can encrypt blocks of 18
bits. Two prime numbers p = 613, q = 431 give n = 264203.

(p-1)(q-1) = 263160

Random number in range 614 -- 263159 inclusive, relatively prime to
263160: d = 200003. This is prime and not a factor of 263160: it will do.

$e = d^{-1}$ mod 263160
\begin{itemize}
\item We have a program to compute this: the Euclidean algorithm computes the greatest common divisor or factor of two numbers, and expresses it as a linear combination of the two numbers.
\item Feed 200003 and 263160 into the Euclidean Algorithm application and it says:
  \begin{itemize}
  \item $1 = -68573 \times 263160 + 90227 \times 200003$;
  \item GCD(263160,200003) = 1
  \end{itemize}
\item The second equation tells us something we knew already; the first tells us -
\end{itemize}

The reciprocal or inverse (mod 263160) of d = 200003 is e = 90227
\end{frame}

\begin{frame}
\frametitle{RSA example}
Check with a calculator that $200003 \times 90227$ \% 263160 = 1

We now have a completely defined RSA crytposystem which we can use to encrypt and decrypt data in blocks of 18-bits.
\begin{itemize}
\item The public key is (n = 264203, e = 90227)
\item Each block we interpret as a number $< 2^{18}$ = 262144; we can do arithmetic modulo n with such blocks.
\end{itemize}

Example: to encrypt the block 000100 000101 000100:
\begin{itemize}
\item 000100 000101 000100 (binary) = 16708 (decimal)
\item $16708^{90227}$ \% 264203 = 111885 using modular exponentiation app
\item = 011011 010100 001101
\end{itemize}

Decrypting:
\begin{itemize}
\item $111885^{200003}$ \% 264203 = 16708 = the original bits
\end{itemize}
\end{frame}

\begin{frame}
\frametitle{Security of RSA}
Alice sends Bob a message $m$ using Bob's $n$, $e$ which are public. The computation $c = m^e~\%~n$ is easy and $c$ is sent to Bob.

Bob must use his (private) $d$ to recover m: $m = c^d~\%~n$, also easy given knowledge of $d$.

An eavesdropper must be able to figure out $d$. She knows $n$ and $e$, and knows that $d$ is the reciprocal of $e$ modulo $(p-1)(q-1)$. But she does not know $p, q$, although she does know $n = pq$.

Eve can figure out p, q from pq by \emph{factorisation} if p, q, are small; but interestingly, inferring primes p, q from their product pq is HARD when p, q, are sufficiently large - around 1000 bits, for instance.

As of 2010, the largest factorized pq was 768 bits long. This took, on a grid of parallel CPUs around 1500 CPU years (2 actual years on hundreds of computers). No larger RSA key is known publicly to have been factorized. 

In practice, 1024 to 4096 a considered to suffice.
\end{frame}

\begin{frame}
\frametitle{Further reading}
You will find a number of resources for exploring RSA at 
\texttt{\small\color{blue}http://computing.northumbria.ac.uk/staff/cgmb3/teaching/\\cryptography/index\_crypto.html}

In particular, 
\begin{itemize}
\item \href{http://computing.northumbria.ac.uk/staff/cgmb3/teaching/cryptography/EulerFermatRSA.pdf}{\textcolor{blue}{A short paper}}[click] on the Euler-Fermat theorem and RSA;
\item \href{http://computing.northumbria.ac.uk/staff/cgmb3/teaching/cryptography/MathsOfRSA.pdf}{\textcolor{blue}{An overview}}[click] of the mathematical basis of RSA;
\item \href{http://computing.northumbria.ac.uk/staff/cgmb3/teaching/cryptography/PKCAttacks/BonehRSA-survey.pdf}{\textcolor{blue}{A survey of attacks}}[click] on RSA;
\item \href{http://computing.northumbria.ac.uk/staff/cgmb3/teaching/cryptography/PKCAttacks/KocherTimingAttacks.pdf}{\textcolor{blue}{Another survey of attacks}}[click] on RSA;
\item The Wikipedia articles on Integer factorization and on RSA are a good read.
\end{itemize}
\end{frame}

\end{document}

