%&xelatex
\documentclass[xcolor=svgnames]{beamer}
\usepackage[british]{babel}
\usepackage[T1]{fontenc}
\usepackage[utf8]{inputenc}

\usepackage{tcolorbox,minted}
\usepackage{graphicx}
\usepackage{booktabs}

%\usetheme[block=fill,progressbar=frametitle]{metropolis}
\usepackage{lmodern}
\usepackage{textgreek}
\usepackage{bytefield}
\usepackage{tikz}
\usetikzlibrary{shapes.misc, shapes.symbols, positioning, calc}
\usetikzlibrary{arrows.meta, shapes.geometric}

\usepackage{xcolor}

\usepackage{siunitx}
\usepackage{tikz-timing}
\usepackage[european]{circuitikz}

\useinnertheme{default}
\useoutertheme{infolines}
\usecolortheme{seahorse}
\setbeamercolor*{alerted text}{fg=blue!75!black}
\setbeamertemplate{blocks}[rounded]
\setbeamertemplate{itemize item}[triangle]
\setbeamertemplate{itemize subitem}[circle]
\setbeamertemplate{itemize subsubitem}[square]

\usecolortheme{rose}
\definecolor{NUblue}{RGB}{62,141,165}
\definecolor{NUbluedark}{RGB}{40,119,143}
\setbeamercolor*{palette primary}{use=structure,fg=white,bg=NUblue}
\setbeamercolor*{palette quaternary}{fg=white,bg=NUbluedark}
\setbeamercolor{section in head/foot}{fg=white,bg=NUbluedark}
\setbeamercolor{subsection in head/foot}{fg=white,bg=NUblue}
\setbeamercolor{frametitle}{fg=NUbluedark!150,bg=NUblue!40}
\setbeamercolor{title in head/foot}{fg=white,bg=NUblue}
\setbeamercolor{author in head/foot}{fg=white, bg=NUbluedark}
\setbeamercolor{date in head/foot}{fg=white, bg=NUblue!60}
\setbeamercolor{title}{fg=NUbluedark!150,bg=NUblue!30}
\setbeamercolor{date}{fg=NUbluedark!150}
\setbeamercolor{block title}{fg=white,bg=NUblue}

\title{UDP, Data formats, Buffers, and Strings}
\subtitle{Control systems and Computer Networks}

\author{Dr Alun Moon}
\date{Lecture 7.2}

\tikzset{onslide/.code args={<#1>#2}{%
\only<#1>{\pgfkeysalso{#2}} % \pgfkeysalso doesn't change the path
}}
\tikzset{temporal/.code args={<#1>#2#3#4}{%
\temporal<#1>{\pgfkeysalso{#2}}{\pgfkeysalso{#3}}{\pgfkeysalso{#4}} % \pgfkeysalso doesn't change the path
}}


\begin{document}
\frame{\maketitle}

\begin{frame}[fragile]{UDP reception}
\begin{itemize}
    \item Send and Recieve functions are \emph{blocking}
    \item Need to run in a \emph{Thread} concurrently with other actions
    \item \texttt{buffer} is written into
    \item pointer to \texttt{SocketAddress}, for sending port and ip address
\end{itemize}
\begin{exampleblock}{}
\begin{minted}{c}
while(1){
    char buffer[1024]; /* 1k bytes */
    SocketAddress source;
    int len = udp.recvfrom( &source,
                             buffer, sizeof(buffer));
    buffer[len]='\0';
\end{minted}
\end{exampleblock}
\end{frame}

\begin{frame}[fragile]{Common Data Format}
\begin{itemize}
    \item Line oriented
    \item Key:Value pairs
\end{itemize}
\begin{block}{}
\begin{itemize}
    \item Email headers RFC822 \url{https://tools.ietf.org/html/rfc822}
    \item HTTP headers RFC7230 \url{https://tools.ietf.org/html/rfc7230}
\end{itemize}
\end{block}

\begin{exampleblock}{}
\begin{verbatim}
Date: 26/02/2018
Pot1: 0.267
AccX: -0.01
\end{verbatim}
\end{exampleblock}

Two tasks:
\begin{enumerate}
    \item Split buffer into lines
    \item Split line into \emph{Key}:\emph{Value} pairs
\end{enumerate}

\end{frame}

\begin{frame}[fragile]{UDP Data}{Array of bytes}
    \begin{itemize}
        \item UDP Datgrams hold the data as an array of bytes
\begin{minted}{java}
    byte[] payload;
    DatagramPacket packet( payload, payload.length );
\end{minted}

\item Converting Bytes to Strings
\begin{minted}{java}
    String message = new String( packet.getData() );
\end{minted}

\item Converting Strings to bytes
\begin{minted}{java}
    packet.setData( message.getBytes() );
\end{minted}

\end{itemize}
\end{frame}

\begin{frame}[fragile]{Splitting the input}
Java Strings have all you need for handling the message data,
\begin{itemize}
    \item \texttt{split} returns an array of strings
\begin{minted}{java}
    String[] lines = message.split("\n");
\end{minted}

\item \texttt{trim} gets rid of whitespace at the beginning and end of a string
\begin{minted}{java}
    String clean = message.trim();
    String lines[] = message.trim().split("\n");
\end{minted}

\item arrays of lines can be iterated over
\begin{minted}{java}
    for( String line : lines ) {...}
\end{minted}

\end{itemize}
\end{frame}

\begin{frame}[fragile]{Handling lines}
\begin{itemize}
\item lines can be split on a colon delimeter
\begin{minted}{java}
    String[] pair = line.split(":");
\end{minted}

\item there should be 2 elements in the array, the first is the key.
\begin{minted}{java}
    String key = pair[0];
\end{minted}

\item  the second is the value.
\begin{minted}{java}
    String value = pair[1];
\end{minted}

\item I use a hashtable to store the key-value pairs
\end{itemize}
\end{frame}

\begin{frame}[fragile]{Java -- String methods}
\begin{minted}{java}
DatagramPacket msg = new DatagramPacket(buffer, buffer.length);
socket.receive(msg);
String message = new String(msg.getData());
String[] lines = message.trim().split("\n");
for(String l : lines ) {
    if( l.length()>0 ) {
        String[] pair = l.trim().split(":");
        if(pair.length==2) datatable.put(pair[0], pair[1]);
    }
}
\end{minted}
\end{frame}


\begin{frame}[fragile]{C String handling is not quite so neat}
A Quick review of strings in C:
\begin{itemize}
    \item Strings are arrays of \mintinline{c}{char}
    \begin{minted}{c}
        char string[80];
    \end{minted}
    \item Pointers to \mintinline{c}{char} are also strings
    \begin{minted}{c}
        char *string;
    \end{minted}
    \item Strings are stored as \textsc{ascii} values\\ with a terminating byte value of zero
    \begin{minted}{c}
        "hello" === {104,101,108,108,111,0}
    \end{minted}
    \item Remember \alert{No Bounds Checks on Arrays}
\end{itemize}

\end{frame}

\begin{frame}[fragile]{UDP Datagrams}
The MBED library just sends a block of data bytes/char
\begin{exampleblock}{}
\begin{minted}{c}
    UDPSocket udp;
    char *buffer;
    int data_size;
    udp.sendto( server, buffer, data_size );
\end{minted}
\end{exampleblock}
How to find size of data?  Numer of bytes to send?
\begin{itemize}
    \item Use size of array declaration from compile time
    \begin{minted}{c}
        char data[256];
        int len = sizeof(data);
    \end{minted}
    \mintinline{c}{sizeof} is a compile time operator
    \item length of a string
    \begin{minted}{c}
        int len = strlen( datamessage );
    \end{minted}
\end{itemize}
\end{frame}

\begin{frame}[fragile]{Building a Datagram message}
For simple messages
\begin{exampleblock}{Print to string}
\begin{minted}{c}
    sprintf( buffer, "pot:%f \n", value);
\end{minted}
\end{exampleblock}

Slightly more complex
\begin{exampleblock}{Print to string}
\begin{minted}{c}
    sprintf( buffer, "pot 1:%f \npot 2: %f\n", one, two);
\end{minted}
\end{exampleblock}

Does it scale?
\end{frame}

\begin{frame}[fragile]{More complex messges}{build it a line at a time}
\begin{description}
    \item[\mintinline{c}{strcat}] concatenates (joins) strings
    \item[\mintinline{c}{sprintf}] writes formatted text to strings
\end{description}
Start with empty buffer
\begin{exampleblock}{}
\begin{minted}{c}
  char buffer[512], line[80];
  buffer[0]='\0'; /* make buffer look like empty string */
\end{minted}
\end{exampleblock}
Format each line and concatenate it with buffer
\begin{exampleblock}{}
\begin{minted}{c}
    sprintf(line,"POT 1:%f\n",left.read());
    strcat(buffer,line);

    sprintf(line,"POT 2:%f\n",right.read());
    strcat(buffer,line);

\end{minted}
\end{exampleblock}

\end{frame}

\begin{frame}[fragile]{Splitting incoming messages in C}
    By ``hand''
    \begin{block}{Read line}
    \begin{enumerate}
        \item point line-pointer to beginning of buffer\\
            \hfill\mintinline{c}{line = buffer;}
        \item move the buffer-pointer along, looking for the end of line character \\\hfill\mintinline{c}{while(*buffer!='\n')++buffer;}
        \item overwrite line ending with string terminator\\
        \hfill\mintinline{c}{*buffer='\0';}
        \item move buffer-pointer to first character of next line\\
        \hfill\mintinline{c}{buffer++;}
    \end{enumerate}
    \end{block}
    \begin{exampleblock}{Steps 3 and 4 can be combined}
        \hfill\mintinline{c}{*buffer++ = '\0'}
    \end{exampleblock}
\end{frame}

\begin{frame}[fragile]{C String library to the rescue}
The C string library has a function that performs a similar task \mintinline{c}{strtok}

\begin{block}{Using \mintinline{c}{strtok}}
\begin{enumerate}
    \item In the initial call to \mintinline{c}{strtok} supply a pointer to the initial buffer, and the list (string) of delimiters\\
    \hfill\mintinline{c}{char line = strtok(buffer, "\n\r");}
    \item For subsequent calls, pass a NULL pointer to \mintinline{c}{strtok}, it ``remembers'' where it is in the original buffer.\\
    \hfill\mintinline{c}{ line = strtok(NULL, "\n\r");}
\end{enumerate}
\end{block}

\begin{alertblock}{Nested \mintinline{c}{strtok}}
Because \mintinline{c}{strtok} remembers where it is internally, it cant be nested, one for reading lines, and one for splitting key value pairs.
\end{alertblock}
\end{frame}

\begin{frame}[fragile]{Two solutions }
\begin{block}{Use delimeter changes}
    \begin{enumerate}
        \item find first \emph{key} token \hfill\mintinline{c}{key = strtok(buffer,":");}
        \item \emph{value} is the rest of the line
        \hfill\mintinline{c}{value = strtok(NULL,"\n ");}
        \item next \emph{key} is
        \hfill\mintinline{c}{key = strtok(NULL,":");}
        \item \emph{value} is the rest of the line
        \hfill\mintinline{c}{value = strtok(NULL,"\n ");}
    \end{enumerate}
\end{block}

\begin{exampleblock}{Re-entrant \mintinline{c}{strtok_r}}
There is a version of \mintinline{c}{strtok}, that uses an external parameter to remember its place, these can be nested.

\begin{enumerate}
    \item Read line \hfill\mintinline{c}{line = strtok_r(buffer, "\n", &loc);}
    \begin{itemize}
        \item split line \hfill\mintinline{c}{key = strtok(line,":");}
        \item  \hfill\mintinline{c}{value = strtok(NULL,":");}
    \end{itemize}
    \item read next line
        \hfill\mintinline{c}{line = strtok_r(NULL, "\n", &loc);}
\end{enumerate}
\end{exampleblock}
\end{frame}

\begin{frame}[fragile]{For loop}
\begin{itemize}
    \item \mintinline{c}{strtok} fits nicely (if clumbersome looking) into a \mintinline{c}{for} loop
    \item  \mintinline{c}{strtok} returns \mintinline{c}{NULL} if there are no more tokens to be found.
\end{itemize}

\begin{exampleblock}{}
\begin{minted}{c}
for(
    line = strtok_r(buffer, "\r\n", &nextline);
    line;
    line = strtok_r(NULL, "\r\n", &nextline)
) {
    key = strtok(line, ":");
\end{minted}
\end{exampleblock}
\end{frame}

\begin{frame}[fragile]{Using Key/values}
\begin{itemize}
    \item  Suppose I have a multi-line message.
    \item First key is a message id.
    \item different functions are needed for each message.
    \item Can't use \mintinline{c}{switch} statements
    \item \mintinline{c}{if  else if} with \mintinline{c}{strcmp} is messy
\end{itemize}
\begin{exampleblock}{}
\begin{minted}{c}
if( strcmp(key, "first-id")==0)
else
if (strcmp(key, "second-id")==0 )
else
\end{minted}
\end{exampleblock}
\end{frame}

\begin{frame}[fragile]{Look-up table and function pointers}
\begin{itemize}
    \item action is function that takes pointer to value (assume \mintinline{c}{strtok})\\
    \begin{minted}{c}
        typedef int (*action)(char *value);
    \end{minted}
    \item structure pairing string with action
    \begin{minted}{c}
    typedef struct {
        char *key;
        action act;
    } tableentry;
    \end{minted}
    \item Array of these
    \begin{minted}{c}
        tableentry[] lookup = {
            "foo", function1
            "bar", somethingelse
        };
    \end{minted}
\end{itemize}
\end{frame}

\begin{frame}[fragile]{Tokenise message}
\begin{itemize}
    \item First token in buffer is ``action name''
    \begin{minted}{c}
        char *name = strtok(buffer, ":");
    \end{minted}
    \item Second in paramater (may be status)
    \begin{minted}{c}
        char *state = strtok_r(NULL, '\n');
    \end{minted}
    \item Find entry in lookup table with name
    \begin{minted}{c}
        int n = findintable(lookup, name);
    \end{minted}
    \item Call function
    \begin{minted}{c}
        lookup[n].act(state);
    \end{minted}
    \item Subsequent calls to \mintinline{c}{strtok(NULL,...)} in ``action'' walk down buffer  until exhausted.
\end{itemize}

\end{frame}

\begin{frame}[fragile]{Table lookup made easy}
\begin{itemize}
    \item Provide a comparison function with the same semantics as \texttt{strcmp} ie for \texttt{compare(A,B)}
    \begin{tabular}{rl}
        -1 & A is before B \\
        0 & A and B are equal in order \\
        1 & A is after B
    \end{tabular}
    \begin{minted}{c}
        int order(void *A, void *B) {
            tableentry *a=A , *b=B;
            return strcmp(a->name, b->name);
        }
    \end{minted}
    \item sort the table into order.
    \begin{minted}{c}
qsort( lookup, /* array of entries */
       sizeof(lookup)/sizeof(tableentry), /* number of entries */
       sizeof(tableentry),  /* size of each entry */
       order /* ordering function */
   );
    \end{minted}
\end{itemize}
\end{frame}

\begin{frame}[fragile]{Find the entry}
\begin{itemize}
\item create a placeholder
\begin{minted}{c}
tableentry key;
\end{minted}
\item set the name to the key \mintinline{c}{char *key}
\begin{minted}{c}
key.name = key;
\end{minted}
\item call search
\begin{minted}{c}
tableentry *result = bsearch( &key, /* pointer to key */
                lookup, /* array of entries */
                sizeof(lookup)/sizeof(tableentry), /* number of entries */
                sizeof(tableentry),  /* size of each entry */
                order /* ordering function */
       );
\end{minted}
\item call the function from the result
\begin{minted}{c}
result->act(state);
\end{minted}
\end{itemize}
\end{frame}

\end{document}
