%&xelatex
\documentclass[xcolor=svgnames]{beamer}
\usepackage[british]{babel}
\usepackage[T1]{fontenc}
\usepackage[utf8]{inputenc}

\usepackage{tcolorbox,minted}
\usepackage{graphicx}
\usepackage{booktabs}

%\usetheme[block=fill,progressbar=frametitle]{metropolis}
\usepackage{lmodern}
\usepackage{textgreek}
\usepackage{bytefield}
\usepackage{tikz}
\usetikzlibrary{shapes.misc, shapes.symbols, positioning, calc}
\usetikzlibrary{arrows.meta, shapes.geometric}

\usepackage{xcolor}

\usepackage{siunitx}
\usepackage{tikz-timing}
\usepackage[european]{circuitikz}

\useinnertheme{default}
\useoutertheme{infolines}
\usecolortheme{seahorse}
\setbeamercolor*{alerted text}{fg=blue!75!black}
\setbeamertemplate{blocks}[rounded]
\setbeamertemplate{itemize item}[triangle]
\setbeamertemplate{itemize subitem}[circle]
\setbeamertemplate{itemize subsubitem}[square]

\usecolortheme{rose}
\definecolor{NUblue}{RGB}{62,141,165}
\definecolor{NUbluedark}{RGB}{40,119,143}
\setbeamercolor*{palette primary}{use=structure,fg=white,bg=NUblue}
\setbeamercolor*{palette quaternary}{fg=white,bg=NUbluedark}
\setbeamercolor{section in head/foot}{fg=white,bg=NUbluedark}
\setbeamercolor{subsection in head/foot}{fg=white,bg=NUblue}
\setbeamercolor{frametitle}{fg=NUbluedark!150,bg=NUblue!40}
\setbeamercolor{title in head/foot}{fg=white,bg=NUblue}
\setbeamercolor{author in head/foot}{fg=white, bg=NUbluedark}
\setbeamercolor{date in head/foot}{fg=white, bg=NUblue!60}
\setbeamercolor{title}{fg=NUbluedark!150,bg=NUblue!30}
\setbeamercolor{date}{fg=NUbluedark!150}
\setbeamercolor{block title}{fg=white,bg=NUblue}

\title{Threads, Events}
\subtitle{Control systems and Computer Networks}

\author{Dr Alun Moon}
\date{Lecture 6.2}

\tikzset{onslide/.code args={<#1>#2}{%
\only<#1>{\pgfkeysalso{#2}} % \pgfkeysalso doesn't change the path
}}
\tikzset{temporal/.code args={<#1>#2#3#4}{%
\temporal<#1>{\pgfkeysalso{#2}}{\pgfkeysalso{#3}}{\pgfkeysalso{#4}} % \pgfkeysalso doesn't change the path
}}


\begin{document}
\frame{\maketitle}

\begin{frame}[fragile]{Threads}
   MBED Threads look a lot like the POSIX threads you've seen.
\begin{tcolorbox}
\begin{minted}{c}
Thread worker;
worker.start( flash_red );

void flash_red(void) { while(1){  } }
\end{minted}
\end{tcolorbox}
\begin{itemize}
    \item each thread has it's own loop
    \item \mintinline{c}{while{1}} means the loop and thread keep going forever.
    \item functions like \mintinline{c}{join} exist
\end{itemize}
\end{frame}

\begin{frame}[fragile]{Events and Dispatch}
\begin{itemize}
    \item MBED Events are handled by \texttt{EventQueue}.
    \item Events can be generated by libraries for devices, or programmatically.
    \item Events are \emph{dispatched} to their handlers
    \item The \texttt{EventQueue} can dispatch its events for a given length of time, or continuously
    \item The \emph{dispatch} functions return when finished
    \begin{itemize}
        \item For continuous operation the \texttt{EventQueue} needs to be in its own thread.
    \end{itemize}
\end{itemize}
\begin{tcolorbox}
\begin{minted}{c}
Thread worker;
EventQueue queue ;

worker.start(callback(&queue,
             &EventQueue::dispatch_forever ));
\end{minted}
\end{tcolorbox}

\end{frame}

\begin{frame}[fragile]{Periodic events}
Remember the problem of working out the timing of loops using \texttt{wait}:
\begin{itemize}
    \item If I want a loop at a particular period
    \item I have to use a \texttt{wait} time that takes into account the execution time of the code (which might vary considerably)
\end{itemize}

We can register events to be triggered at a periodic rate
\begin{tcolorbox}
\begin{minted}{c}
void blink(void){
    green = !green;
}

queue.call_every(300, blink);
\end{minted}
\end{tcolorbox}
\begin{description}
    \item[Note:] the event function does not need a \mintinline{c}{while(1)} loop,\\
    it is called \emph{once} at each period.
\end{description}
\end{frame}

\begin{frame}[fragile]{Events and Interrupts}
Recall \emph{Interrupt Service Routines} (\emph{ISR}) cannot perform complex or lengthy operations, such as serial or networks communications.

\begin{itemize}
    \item An ISR can trigger an event
    \item The event is handled in the context of the \emph{event-loop} outside of the ISR.
\end{itemize}
\begin{tcolorbox}
\begin{minted}{c}
void blink(void){
    pc.printf("This is not in an ISR so I can do long (time) operations\n");
}

Thread worker;
EventQueue queue ;

InterruptIn sw(SW2);
sw.fall(queue.event(blink));
\end{minted}
\end{tcolorbox}
\end{frame}

\end{document}
