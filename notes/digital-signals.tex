%&xelatex
\documentclass[xcolor=svgnames]{beamer}
\usepackage[british]{babel}
\usepackage[T1]{fontenc}
\usepackage[utf8]{inputenc}

\usepackage{tcolorbox}
\usepackage{graphicx}
\usepackage{booktabs}

%\usetheme[block=fill,progressbar=frametitle]{metropolis}
\usepackage{lmodern}

\usepackage{bytefield}
\usepackage{tikz}
\usetikzlibrary{shapes.misc, shapes.symbols, positioning, calc}

\usepackage{xcolor}

\usepackage{siunitx}
\usepackage{tikz-timing}
\usepackage{circuitikz}

\useinnertheme{default}
\useoutertheme{infolines}
\usecolortheme{seahorse}
\setbeamercolor*{alerted text}{fg=blue!75!black}
\setbeamertemplate{blocks}[rounded]
\setbeamertemplate{itemize item}[triangle]
\setbeamertemplate{itemize subitem}[circle]
\setbeamertemplate{itemize subsubitem}[square]

\usecolortheme{rose}
\definecolor{NUblue}{RGB}{62,141,165}
\definecolor{NUbluedark}{RGB}{40,119,143}
\setbeamercolor*{palette primary}{use=structure,fg=white,bg=NUblue}
\setbeamercolor*{palette quaternary}{fg=white,bg=NUbluedark}
\setbeamercolor{section in head/foot}{fg=white,bg=NUbluedark}
\setbeamercolor{subsection in head/foot}{fg=white,bg=NUblue}
\setbeamercolor{frametitle}{fg=NUbluedark!150,bg=NUblue!40}
\setbeamercolor{title in head/foot}{fg=white,bg=NUblue}
\setbeamercolor{author in head/foot}{fg=white, bg=NUbluedark}
\setbeamercolor{date in head/foot}{fg=white, bg=NUblue!60}
\setbeamercolor{title}{fg=NUbluedark!150,bg=NUblue!30}
\setbeamercolor{date}{fg=NUbluedark!150}
\setbeamercolor{block title}{fg=white,bg=NUblue}

\title{Digital Signals}
\subtitle{Control systems and Computer Networks}

\author{Dr Alun Moon}
\date{Lecture 1.3}
\begin{document}
\frame{\maketitle}

\begin{frame}{Digital Signals}{What is a digital signal?}

\vspace*{\fill}
A Digital Signal is:\pause\\
\vspace{\fill}
\begin{tabular}{cc}
\onslide<+->{True} & \onslide<+->{False} \\
\onslide<+->{1} & \onslide<+->{0} \\
\onslide<+->{on} & \onslide<+->{off} \\
\onslide<+->{Pressed} & \onslide<+->{Not-pressed} \\
\onslide<+->{High} & \onslide<+->{Low} \\
\onslide<+->{\SI{5}{V}} & \onslide<+->{\SI{0}{V}} \\
\onslide<+->{\SI{3.3}{V}} & \onslide<+->{\SI{0}{V}}
\end{tabular}
\vspace{\fill}
\begin{itemize}[<+->]
  \item from a software perspective anything convenient for us to use
  \item there are external limitations and constraints,
  \begin{itemize}
    \item Physics
    \item Standards
  \end{itemize}
\end{itemize}

\end{frame}

\begin{frame}{Electrical Characteristics}
  Generally :
  \begin{description}
    \item[positive voltage] logical 1
    \item[negative voltage] logical 0
  \end{description}
\vspace{\fill}\pause
Specific technologies have specific voltages for \alert{\emph{on}}
\begin{description}
  \item[TTL] \alert Transistor \alert Transistor \alert Logic \quad \SI{5}{V}
  \item[CMOS] \alert Complementary \alert Metal \alert Oxide \alert Semiconductor
  \quad \SI{3.3}{V}
\end{description}

\end{frame}

\begin{frame}{Sequences}
  Digital signals exist in sequences\ldots\pause
  \begin{itemize}[<+->]
    \item Traffic Lights
    \begin{itemize}
        \item Red $\rightarrow$ Red,Amber $\rightarrow$ Green
          $\rightarrow$ Amber $\rightarrow$ Red \ldots
    \end{itemize}
    \item Flashing
    \begin{itemize}
      \item  On $\rightarrow$ Off $\rightarrow$ On \ldots
    \end{itemize}
  \end{itemize}

\pause
Can be written as a \alert{Timing Diagram}\\[1em]
\begin{tikztimingtable}
  Red   & HHHHHHLLLLLLHH\\
  Amber & LLLLHHLLLLHHLL\\
  Green & LLLLLLHHHHLLLL\\
\end{tikztimingtable}
\end{frame}

\begin{frame}{Digital IO from the C}
Microcontrollers (C) have dedicated hardware for digital IO.
GPIO

\begin{circuitikz}
\draw (0,0) node[rectangle] {}
  to[short, d-*] (1,0)
  to[R] (2,0);
\end{circuitikz}

\end{frame}

\end{document}
%%%%======================
---
title: Digital Signals
file: digital-signals.tex
lecturer: Dr Alun Moon
---
Introduces the concepts of digital signals:
 * Physical characteristics
 * Programming and hardware support
 * Mathematical models
 * IO Circuits and signals

sequences - traffic lights
flashing - variable and pattern
two handed start

voltage divider-pull up-pulldown
led - current sink - open drain mode


Live demos...
