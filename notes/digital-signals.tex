%&xelatex
\documentclass[xcolor=svgnames]{beamer}
\usepackage[british]{babel}
\usepackage[T1]{fontenc}
\usepackage[utf8]{inputenc}

\usepackage{tcolorbox}
\usepackage{graphicx}
\usepackage{booktabs}

%\usetheme[block=fill,progressbar=frametitle]{metropolis}
\usepackage{lmodern}
\usepackage{textgreek}
\usepackage{bytefield}
\usepackage{tikz}
\usetikzlibrary{shapes.misc, shapes.symbols, positioning, calc}
\usetikzlibrary{arrows.meta}

\usepackage{xcolor}

\usepackage{siunitx}
\usepackage{tikz-timing}
\usepackage[european]{circuitikz}

\useinnertheme{default}
\useoutertheme{infolines}
\usecolortheme{seahorse}
\setbeamercolor*{alerted text}{fg=blue!75!black}
\setbeamertemplate{blocks}[rounded]
\setbeamertemplate{itemize item}[triangle]
\setbeamertemplate{itemize subitem}[circle]
\setbeamertemplate{itemize subsubitem}[square]

\usecolortheme{rose}
\definecolor{NUblue}{RGB}{62,141,165}
\definecolor{NUbluedark}{RGB}{40,119,143}
\setbeamercolor*{palette primary}{use=structure,fg=white,bg=NUblue}
\setbeamercolor*{palette quaternary}{fg=white,bg=NUbluedark}
\setbeamercolor{section in head/foot}{fg=white,bg=NUbluedark}
\setbeamercolor{subsection in head/foot}{fg=white,bg=NUblue}
\setbeamercolor{frametitle}{fg=NUbluedark!150,bg=NUblue!40}
\setbeamercolor{title in head/foot}{fg=white,bg=NUblue}
\setbeamercolor{author in head/foot}{fg=white, bg=NUbluedark}
\setbeamercolor{date in head/foot}{fg=white, bg=NUblue!60}
\setbeamercolor{title}{fg=NUbluedark!150,bg=NUblue!30}
\setbeamercolor{date}{fg=NUbluedark!150}
\setbeamercolor{block title}{fg=white,bg=NUblue}

\title{Digital Signals}
\subtitle{Control systems and Computer Networks}

\author{Dr Alun Moon}
\date{Lecture 1.2}

\tikzset{onslide/.code args={<#1>#2}{%
\only<#1>{\pgfkeysalso{#2}} % \pgfkeysalso doesn't change the path
}}
\tikzset{temporal/.code args={<#1>#2#3#4}{%
\temporal<#1>{\pgfkeysalso{#2}}{\pgfkeysalso{#3}}{\pgfkeysalso{#4}} % \pgfkeysalso doesn't change the path
}}


\begin{document}
\frame{\maketitle}

\begin{frame}{Digital Signals}{What is a digital signal?}

\vspace*{\fill}
A Digital Signal is:\pause\\
\vspace{\fill}
\begin{tabular}{cc}
\onslide<+->{True} & \onslide<+->{False} \\
\onslide<+->{1} & \onslide<+->{0} \\
\onslide<+->{on} & \onslide<+->{off} \\
\onslide<+->{Pressed} & \onslide<+->{Not-pressed} \\
\onslide<+->{High} & \onslide<+->{Low} \\
\onslide<+->{\SI{5}{V}} & \onslide<+->{\SI{0}{V}} \\
\onslide<+->{\SI{3.3}{V}} & \onslide<+->{\SI{0}{V}}
\end{tabular}
\vspace{\fill}
\begin{itemize}[<+->]
  \item from a software perspective anything convenient for us to use
  \item there are external limitations and constraints,
  \begin{itemize}
    \item Physics
    \item Standards
  \end{itemize}
\end{itemize}

\end{frame}

\begin{frame}{Electrical Characteristics}
  Generally :
  \begin{description}
    \item[positive voltage] logical 1
    \item[negative voltage] logical 0
  \end{description}
\vspace{\fill}\pause
Specific technologies have specific voltages for \alert{\emph{on}}
\begin{description}
  \item[TTL] \alert Transistor \alert Transistor \alert Logic \quad \SI{5}{V}
  \item[CMOS] \alert Complementary \alert Metal \alert Oxide \alert Semiconductor
  \quad \SI{3.3}{V}
\end{description}

\end{frame}

\begin{frame}{Sequences}
  Digital signals exist in sequences\ldots\pause
  \begin{itemize}[<+->]
    \item Traffic Lights
    \begin{itemize}
        \item Red $\rightarrow$ Red,Amber $\rightarrow$ Green
          $\rightarrow$ Amber $\rightarrow$ Red \ldots
        \item $\langle \{r\},\{r,a\},\{g\},\{a\}\rangle$
        \item Can be written as a \alert{Timing Diagram}\\
        \begin{tikztimingtable}
          Red   & HHHHHHLLLLLLHH\\
          Amber & LLLLHHLLLLHHLL\\
          Green & LLLLLLHHHHLLLL\\
        \end{tikztimingtable}
    \end{itemize}
    \item Flashing
    \begin{itemize}
      \item  On $\rightarrow$ Off $\rightarrow$ On \ldots
      \item \texttiming{HHLLHHLLHHLLHHLLHHLLHHLLHHLLHHLL}
    \end{itemize}
  \end{itemize}

\end{frame}

\begin{frame}{Digital IO from the \textmu{}C}{%
Microcontrollers (\textmu{}C) have dedicated hardware for digital IO.}
\begin{itemize}
  \item The K64F has 5 ports with 32 IO pins which can be used as GPIO pins
        (\alert General \alert Purpose \alert Input \alert Output)
  \item The IO circuit has a number of configureable options for each pin,
        accessed through several registers
  \item \alert{\textbf{ALL}} the appropriate bits need to be set or it doesn't work.
\end{itemize}

\end{frame}

\begin{frame}{GPIO Hardware Registers}{Sequence and purpose of bits to set}
There are several bits to set to configure the pin
\begin{enumerate}
  \item System Clock Gating Control Register  \alert{SCGC} \\
    Enables the clock signal for the port, making it function
  \item Pin Control Register \alert{PORTx\_PCRn}\\
    a 32bit register for each pin setting several options
    \begin{itemize}
      \item IRQC -- Interrupt configuration (what causes an interrupt to occur)
      \item MUX -- Pins have multiple functions, this selects the function to use.
      \item DSE -- Drive Strength, the electrical characteristics of the output
      \item ODE -- Open Drain, elctrical connections of the Output
      \item PFE -- Passive Filter for inputs (debounce and glitch rejection)
      \item SLE -- Slew Rate, how fast the output switches between high and Low
      \item PE -- enable pull up or down resistor for inputs
      \item PS -- selects the pull-up or pull-down resistor.
    \end{itemize}
\end{enumerate}
\end{frame}

\begin{frame}{GPIO Hardware Registers}{Port Registers}
  Each Port has several registers to use for the actual IO operations.
  Each bit in the register corresponds to an external pin.
  \begin{description}
    \item[GPIOx\_PDOR] Port Data Output Register
    \begin{enumerate}\setcounter{enumi}{-1}
      \item Set the output to logic 0
      \item Set the output to logic 1
    \end{enumerate}
    \item[GPIOx\_PSOR] Port Set Output Register
    \begin{enumerate}\setcounter{enumi}{-1}
      \item output does not change
      \item Set the output to logic 1
    \end{enumerate}
    \item[GPIOx\_PCOR] Port Clear Output Register
    \begin{enumerate}\setcounter{enumi}{-1}
      \item output does not change
      \item Set the output to logic 0
    \end{enumerate}
    \item[GPIOx\_PTOR] Port Toggle Output Register
    \begin{enumerate}\setcounter{enumi}{-1}
      \item Output does not change
      \item Change the logic state of the output
    \end{enumerate}
  \end{description}
\end{frame}
\begin{frame}{GPIO Hardware Registers}{Port Registers}
  \begin{description}
    \item[GPIOx\_PDIR] Port Data Input Register
    \begin{enumerate}\setcounter{enumi}{-1}
      \item Pin is set to input logic 0 (or is not configured)
      \item Pin is set ti input logic 1
    \end{enumerate}
    \item[GPIOx\_PDDR] Port Data Direction Register
    \begin{enumerate}\setcounter{enumi}{-1}
      \item GPIO pin set as input
      \item GPIO pin set as output
    \end{enumerate}
  \end{description}
\end{frame}


\begin{frame}{IO Circuits}{Output}
\begin{itemize}
  \item   The \textmu{}C pin is set to 0 or 1
  \item in the case of the K64F $1\equiv\SI{3.3}{V}$
  \item But what does that do?
  \item It depends on the external circuit.
\end{itemize}

\begin{block}{The LED}
\begin{columns}[onlytextwidth]
  \begin{column}{.3\textwidth}
    \ctikzset{bipoles/length=.8cm}
    \begin{circuitikz}
      \draw (0,0) node[vcc] {\SI{3.3}{V}}
        to[R] (0,-1)
        to[empty led] (0,-2)
        to[short, -o] (-1,-2) node[anchor=south]{IO pin};
    \end{circuitikz}
  \end{column}
  \begin{column}{.3\textwidth}
    \ctikzset{bipoles/length=.8cm}
    \begin{circuitikz}
      \draw (0,0) node[vcc] (vcc) {\SI{3.3}{V}}
        to[R] (0,-1)
        to[led,onslide=<3->{color=red}] (0,-2)
        to[short, -o] (-1,-2) node[anchor=south] (gnd) {$0\onslide<2->{\equiv\SI{0}{V}}$};
        \onslide<3->{\draw[-latex,red] (vcc) to[bend right] (gnd) ;}
        \onslide<3->{\draw (vcc -| gnd)
          node[text width=2cm]
          {\small V across diode $\Rightarrow$ LED lights};}
    \end{circuitikz}
  \end{column}
  \begin{column}{.3\textwidth}
    \ctikzset{bipoles/length=.8cm}
    \begin{circuitikz}
      \draw (0,0) node[vcc] {\SI{3.3}{V}}
        to[R] (0,-1)
        to[empty led] (0,-2)
        to[short, -o] (-1,-2) node[anchor=south]{$1\onslide<4->{\equiv\SI{3.3}{V}}$};
        \onslide<5>\draw[dotted] (vcc) to[bend right] (gnd) ;
        \onslide<3->{\draw (vcc -| gnd)
          node[text width=2cm]
          {\small  0V across diode $\Rightarrow$\\ no light};}
    \end{circuitikz}
  \end{column}
\end{columns}
\end{block}
\end{frame}

\begin{frame}{IO Circuits}{Input}
\begin{itemize}
  \item The \textmu{}C pin is set to 0 or 1 by the external circuit
  \item But what does that mean?
  \item It depends on the external circuit.
\end{itemize}
\begin{columns}[onlytextwidth]
\begin{column}{.45\textwidth}
  \begin{block}{Base Shield Push Buttons}
    \begin{minipage}{1cm}
    \ctikzset{bipoles/length=.8cm}
  \begin{circuitikz}
    \draw (0,0) node[vcc] {\SI{3.3}{V}}
      to[R, -*] (0,-1)
      to[push button] (0,-2) node[ground] {};
    \draw (0,-1) to[short,-o] (0.5,-1) node[anchor=south] {pin};
  \end{circuitikz}
\end{minipage}
\begin{minipage}{4cm}
  \begin{itemize}[<+->]
    \item \alert{Not Pressed} pin is connected to \SI{3.3}{V} (logic 1) through
    \emph{pull-up} resistor
    \item \alert{Pressed} pin is shorted to \SI{0}{V} (logic 0)
  \end{itemize}
\end{minipage}
  \end{block}

\end{column}
\begin{column}{.45\textwidth}
\begin{block}{Upper Shield 5-way switch}
\begin{minipage}{1cm}
    \ctikzset{bipoles/length=.8cm}
  \begin{circuitikz}
    \draw (0,0) node[vcc] {\SI{3.3}{V}}
      to[push button,-*] (0,-1)
      to[R] (0,-2) node[ground] {};
    \draw (0,-1) to[short,-o] (0.5,-1) node[anchor=north] {pin};
  \end{circuitikz}
\end{minipage}
  \begin{minipage}{4cm}
    \begin{itemize}[<+->]
      \item \alert{Not Pressed} pin is connected to \SI{0}{V} (logic 0) through
      \emph{pull-down} resistor
      \item \alert{Pressed} pin is shorted to \SI{3.3}{V} (logic 1)
    \end{itemize}
\end{minipage}
\end{block}

\end{column}
\end{columns}

\end{frame}


\begin{frame}{Some Mathematics}
\begin{itemize}
  \item Digital signals are logic values
  \item Can be modelled using discrete maths and Boolean Algebra.
  \item We need a new notation to indicate the change in state
\end{itemize}

\begin{definition}[Change in state]
  The new state is indicated by the use of a prime.
  \[
    a^\prime = a \oplus 1
  \]
The digital signal $a$ has a new value $a^\prime$,  the XOR of the current state and 1
\\[1em]
\hfill \textit{What does this do?}
\end{definition}
\end{frame}

\begin{frame}{Sequences}
  \begin{itemize}[<+->]
    \item A sequence can be written in a manner similar to a set:
    \[ \langle \{r\}, \{r,a\}, \{g\}, \{a\}, \{r\} \rangle \]
    \item Expressing the change in state as a function of other states:
    \[ r^\prime = \neg r \vee g \vee b \]
    \emph{ recursively in this case}
    \item Iteratively:
    \[ b_{n} = n \in \mathbb{Z}_\mathrm{primes} \]
    \[ g_{n+1} = B_1 \Rightarrow g_{n} \oplus 1 \]
  \end{itemize}

\end{frame}

\begin{frame}{Constraints from Standards}
Now we have an understanding of:
\begin{itemize}
  \item What a digital signal is
  \item How to manipulate them
\end{itemize}
We have a question:
\begin{quote}
What should the signal be?  How should it behave?
\end{quote}
This is where constraints from the application domain,
 and any applicable standards, dictate the operation.
\end{frame}

\begin{frame}{Traffic lights}{we can substitute blue for amber}
  For a 2-way junction we have 2 sets of Lights.
  \[ r_n, a_n, g_n, r_w, a_w, g_w \]\pause
  6 bits $2^6 = 64$ possible combinations\pause
  \begin{itemize}[<+->]
    \item What combinations are allowed?
    \item What are the allowed transitions?
  \end{itemize}
\end{frame}

\begin{frame}{Morse Code}
  \begin{itemize}
    \item Morse code is a Digital signal (on and off)
    \item The sequence of states and their timing is defined by standards\\
     International Morse code Recommendation ITU-R M.1677-1
  \end{itemize}

\begin{exampleblock}{Timing}
\begin{enumerate}
  \item short mark, dot or "dit" : "dot duration" is one time unit long
  \item longer mark, dash or "dah": three time units long
  \item inter-element gap between the dots and dashes within a character: one dot duration or one unit long
  \item short gap (between letters): three time units long
  \item medium gap (between words): seven time units long
\end{enumerate}
\end{exampleblock}
\end{frame}

\begin{frame}{Two Handed start}
\begin{itemize}
  \item In many industrial automation environments ``two-handed'' start is used as a safety feature
  \item to start a machine, two start switches, \textbf{must} be pressed together, in order to start.
\end{itemize}
\begin{exampleblock}{Logic}
  \begin{tabular}{rl}
      $m$ & state of motor \\
      $b_1$ & one start switch \\
      $b_2$ & second start switch \\ \midrule
      Constraint & $ b_1 \wedge b_2 \Rightarrow m $ \\ \midrule
      Action & $ m^\prime = m\vee (b_1\wedge b_2) $ \\
  \end{tabular}



\end{exampleblock}
\end{frame}

\begin{frame}{Pullup and pulldown resistors}
\begin{itemize}
  \item  In a number of places pull-up and pull-down resistors are mentioned.
  \item The problem occurs when a wire/pin has no input driving it, what logic value does it have?
  \ctikzset{bipoles/length=.8cm}
  \tikz \draw (0,0) to[short,o-] (0.5,0) to[push button] (2,0);
\end{itemize}

\begin{columns}[onlytextwidth]
\begin{column}{0.48\textwidth}
  \begin{block}{Pull-up resistor}
    \alert{Pull-up} resistor connects the pin to the supply (logic 1) ensuring a 1 when there is no other signal.
    \ctikzset{bipoles/length=.8cm}
  \tikz{
    \draw(0,0) to[short,o-*]  (0.5,0) to[push button] (2,0) node[ground] {};
    \draw(0.5,0) to[R] (0.5,1) node[vcc] {};
    }
  \end{block}
\end{column}
\begin{column}{0.48\textwidth}
  \begin{block}{Pull-down resistor}
    \alert{Pull-down} resistor connects the pin to the ground (logic 0) ensuring a 0 when there is no other signal.
    \ctikzset{bipoles/length=.8cm}
  \tikz{
    \draw(0,0) to[short,o-*] (0.5,0) to[push button] (2,0) node[vcc]{};
    \draw(0.5,0) to[R] (0.5,-1) node[ground]{};
    }

  \end{block}
\end{column}
\end{columns}
\end{frame}

\end{document}

%%%%======================
---
title: Digital Signals
file: digital-signals.tex
lecturer: Dr Alun Moon
---
Introduces the concepts of digital signals:
 * Physical characteristics
 * Programming and hardware support
 * Mathematical models
 * IO Circuits and signals

sequences - traffic lights
flashing - variable and pattern
two handed start

voltage divider-pull up-pulldown
led - current sink - open drain mode


Live demos...
