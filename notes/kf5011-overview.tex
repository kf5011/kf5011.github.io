%&xelatex
\documentclass[xcolor=svgnames]{beamer}
\usepackage[british]{babel}
\usepackage[T1]{fontenc}
\usepackage[utf8]{inputenc}

\usepackage{tcolorbox}
\usepackage{graphicx}
\usepackage{booktabs}

\usetheme[block=fill,progressbar=frametitle]{metropolis}
\usepackage{lmodern}

\usepackage{bytefield}

\usepackage{tikz}
\usetikzlibrary{shapes.misc, shapes.symbols, positioning, calc}

\usepackage{xcolor}
\usepackage{siunitx}

\title{Control systems and Computer Networks}
\subtitle{Module Overview}

\author{Dr Alun Moon}
\date{Lecture 1.1}
\begin{document}
\frame{\maketitle}

\begin{frame}{What will I learn?}
On this module you will learn about computer networks and networked embedded control systems. Many systems now use a network of small
    computers and devices, from modern cars to the Internet of Things. This module looks at the interaction between devices (sensors and
    actuators) and the controlling computer. The various networking and communications technologies that connect these systems together, their
    infrastructure and protocols, will be discussed while considering cybersecurity architecture and operations as appropriate.

    \hspace*{\fill}\emph{from the Module Specification}
\end{frame}
\begin{frame}{What will I learn?}
    You will learn how to control hardware attached to computers. This involves understanding the basics of the electronic circuits used in
    attaching the hardware, the features of the processors that interact with hardware and the writing of device drivers (the code that controls the
    devices). You will learn about microprocessor organisation with respect to Input / Output (I/O) and software development for networked
    control systems using an appropriate language.

    \hspace*{\fill}\emph{from the Module Specification}
\end{frame}

\begin{frame}{Teaching Team}
\begin{tabular}{ll}
Module Tutor & Dr Alun Moon  \\
    & Dr Michael Brockway \\
\end{tabular}
\end{frame}

\begin{frame}{Teaching Structure}
  \begin{description}
    \item[Lectures] $2 \times \SI{1}{\hour}$ lectures
    \begin{itemize}
      \item roughly split into 2 interweaving streams
      \item will try to keep step with labs (no promises)
    \end{itemize}
    \item[Labs/Seminars] $1\times\SI{2}{\hour}$ sessions
    \begin{itemize}
      \item Main practical work
      \item Largely self-contained
    \end{itemize}
    \item[Directed Study] Where appropriate I'll point you in the direction of resources to read
    \item[Independent Learning] A further \SI{9}{\hour}
\end{description}
\end{frame}

\begin{frame}[allowframebreaks]{Teaching Plan}%
\vspace{-1.1em}\begin{enumerate}[Week 1]
    \item \begin{description}
        \item[Seminar] Digital I/O
        \item[Lecture 1] Introduction and Overview
        \item[Lecture 2] Digital Signals
    \end{description}
    \item  \begin{description}
        \item[Seminar] Events and interrupts
        \item[Lecture 1] Memory mapped I/O
        \item[Lecture 2] Discrete time \& Interrupts
    \end{description}
    \item  \begin{description}
        \item[Seminar] Analogue I/O
        \item[Lecture 1] I2C, SPI, CAN, busses
        \item[Lecture 2] Analogue Signals
    \end{description}
    \item  \begin{description}
        \item[Seminar] LCD \& serial comms
        \item[Lecture 1] Display code
        \item[Lecture 2] Sensors
    \end{description}
    \item  \begin{description}
        \item[Seminar] Control \& feedback
        \item[Lecture 1] Control of systems
        \item[Lecture 2] Finite state machines
    \end{description}
    \item  \begin{description}
        \item[Seminar] Assembler and ABI
        \item[Lecture 1] Sampling \& Nyquist limit
        \item[Lecture 2] ARM architecture and assembler, ABI,
    \end{description}
    \item  \begin{description}
        \item[Seminar] Data logging
        \item[Lecture 1] Formats, XML, Json, Yaml
        \item[Lecture 2] TCP stack, RFC1122, RFC1123, RFC768
    \end{description}
    \item  \begin{description}
        \item[Seminar] ?
        \item[Lecture 1] Server logging and analysis
        \item[Lecture 2] Application layer, RFC2616, RFC7540, RFC1123
    \end{description}
    \item  \begin{description}
        \item[Seminar] ?
        \item[Lecture 1] Authentication, validity, md5 hashes, sha/NIST FIPS 180-4
        \item[Lecture 2] TLS RFC5246, SSH / HTTPS RFC2818
    \end{description}
    \item  \begin{description}
        \item[Seminar] ?
        \item[Lecture 1] Encryption
        \item[Lecture 2]  PGP RFC4880
    \end{description}
    \item  \begin{description}
        \item[Seminar] Assignment support
        \item[Lecture 1] Assignment support
        \item[Lecture 2] Assignment support
    \end{description}
\end{enumerate}
\end{frame}

\begin{frame}{Learning Outcomes}
\textbf{Knowledge \& Understanding:}
\begin{itemize}
    \item  Demonstrate knowledge and critical
    understanding of the interaction
    between physical systems, computer
    hardware and software, including
    control theories, network protocols, and
    cybersecurity architecture and
    operations
    \item Apply principles of design and
    implementation of stack models,
    network protocols, control systems, and
    security
\end{itemize}
\end{frame}

\begin{frame}{Learning Outcomes}
\textbf{Intellectual / Professional skills \& abilities:}
\begin{itemize}
    \item Design, implement, test, document and
    evaluate a networked embedded control
    system
    \item Apply software development tools and
    best practice to produce, test and
    debug software for small networked
    control systems using specifications for
    embedded devices and associated
    hardware
\end{itemize}

\end{frame}

\begin{frame}{Learning Outcomes}
\textbf{Personal Values Attributes (Global / Cultural
awareness, Ethics, Curiosity) (PVA):}
\begin{itemize}
    \item Demonstrate independent, critical and
    reflective thinking and practice in the
    development of a networked control
    system, and engagement with appropriate professional and technical
    literature to support and communicate
    such development
\end{itemize}
\end{frame}

\begin{frame}{Software tools and resources}

\end{frame}

\end{document}
%%%%======================
